%        File: hw4.tex
%     Created: Tue Oct 11 01:00 PM 2016 C
% Last Change: Tue Oct 11 01:00 PM 2016 C
%

\documentclass[a4paper]{article}

\title{Math 8640 HW \#4}
\date{10/14/16}
\author{Trevor Steil}

\usepackage{amsmath}
\usepackage{amsthm}
\usepackage{amssymb}
\usepackage{esint}

\newtheorem{theorem}{Theorem}[section]
\newtheorem{corollary}{Corollary}[section]
\newtheorem{proposition}{Proposition}[section]
\newtheorem{lemma}{Lemma}[section]
\newtheorem*{claim}{Claim}
\newtheorem*{problem}{Problem}
%\newtheorem*{lemma}{Lemma}
\newtheorem{definition}{Definition}[section]

\newcommand{\R}{\mathbb{R}}
\newcommand{\N}{\mathbb{N}}
\newcommand{\C}{\mathbb{C}}
\newcommand{\Z}{\mathbb{Z}}
\newcommand{\supp}[1]{\mathop{\mathrm{supp}}\left(#1\right)}
\newcommand{\lip}[1]{\mathop{\mathrm{Lip}}\left(#1\right)}
\newcommand{\curl}{\mathrm{curl}}
\newcommand{\la}{\left \langle}
\newcommand{\ra}{\right \rangle}
\renewcommand{\vec}[1]{\mathbf{#1}}

\newenvironment{solution}{\emph{Solution.}}

\begin{document}
\maketitle

\begin{enumerate}
  \item (Exercies 2.1.6 in Grafakos) Let $M_s(f)(x)$ be the supremum of the average of $|f|$ over all rectangles with sides parallel to the axes,
    which contain $x$. The operator $M_s$ is called the \textit{strong maximal function}.

    \begin{enumerate}
      \item Prove that $M_s$ maps $L^p$ to itself for $p>1$.
      \item Prove that $M_s$ is not weak type (1,1).
    \end{enumerate}

    \begin{proof}
      \begin{enumerate}
        \item
          We already know that the Hardy-Littlewood maximal function maps $L^p$ to $L^p$ for $p>1$. That is, we know $\| M(f) \|_p \leq C \| f \|_p$
          for $f \in L^p$. We will use this bound to prove the strong maximal function maps $L^p$ to $L^p$. We will assume we are working in $\R^2$.
          We can apply the argument inductively to get the result for higher dimensions.

          Let $f \in L^p(\R^2)$. We will use $I_1$ and $I_2$ to denote intervals in $\R$. We then have
          \begin{align*}
            \| M(f) \|_p^p &= \sup_{I_1, I_2} \frac{1}{|I_1| |I_2|} \int_{I_2}^{} \int_{I_1}^{} |f(x_1, x_2)|^p dx_1 dx_2 \\
            &\leq \sup_{I_2} \frac{1}{|I_2|} \int_{I_2}^{} \left( \sup_{I_1} \frac{1}{|I_1|} \int_{I_1}^{} |f(x_1, x_2)|^p dx_1 \right) dx_2 \\
          \end{align*}

          For almost everywhere $x_2 \in \R$, the function $f(x_1, x_2) \in L^p(\R)$. Therefore, we can apply the one-dimensional estimate to get
          \[ \| M(f) \|_p^p \leq \sup_{ I_2 } \frac{1}{|I_2|} \int_{I_2}^{} C \| f( \cdot, x_2) \|_p^p dx_2 .\]
          Because $f \in L^p(\R^2)$, $f(\cdot, x_2) \in L^p$ and we can apply the one-dimensional estimate again to get
          \[ \| M(f) \|_p^p \leq C^2 \|f\|_p^p .\]

        \item

          Let $f_n$ be the characteristic function of a rectangle of width $n$ and height $\frac{1}{n}$ centered at the origin. Then $\| f \|_1 = 1$.
          Take a point $(x_1, x_2)$ with $|x_2| \leq \frac{1}{2n}$ and $x_1 > 0$.. Then by taking a rectangle of height $\delta$ that contains $(x_1, x_2)$ and
          extends to the origin, we get
          \begin{align*}
            Mf(x_1,x_2) &= \sup_{R} \frac{1}{\mu(R)} \int_{R}^{} f d\mu \\
            &\geq \frac{1}{\delta x_1} \frac{n}{2} \delta \\
            &= \frac{n}{2 x_1}
          \end{align*}

          For a fixed $\lambda>0$, we then have $Mf(x_1,x_2) \geq \lambda$ if $x_1 \leq \frac{n}{2 \lambda}$. Therefore,
          \begin{align*}
            %\mu \left( \{ M_s f \geq \lambda \} \right) \geq 2 \mu \left( \{ (x_1, x_2) :
          \end{align*}<++>

      \end{enumerate}

    \end{proof}

  \item (Exercise 1.1.12 in Grafakos) Let $f_1, \dots, f_N \in L^{p,\infty}(X,\mu)$. Prove that
    \[ \left\| \sum_{j=1}^N f_j \right\|_{p,\infty} \leq N \sum_{j=1}^N \| f_j \|_{p,\infty} .\]

    \begin{proof}

      \begin{align*}
        \left\| \sum_{i=1}^N f_i \right\|_{p,\infty} &= \sup \left\{ \lambda d_{\sum_i f_i}(\lambda)^{\frac{1}{p}} : \lambda>0 \right\} \\
        &\leq \sup \left\{ \lambda \left( \sum_{i=1}^N d_{f_i} \left( \frac{\lambda}{N} \right) \right)^{\frac{1}{p}} : \lambda>0 \right\} \\
        &\leq \sup \left\{ \lambda \sum_{i=1}^N d_{f_i} \left( \frac{\lambda}{N} \right)^{\frac{1}{p}} : \lambda >0 \right\} \quad \parbox{4cm}{for $p \geq 1$} \\
        &= N \sup \left\{ \frac{\lambda}{N} \sum_{i=1}^N d_{f_i} \left( \frac{\lambda}{N} \right)^\frac{1}{p} : \lambda > 0 \right\} \\
        &\leq N \sum_{i=1}^N \sup \left\{ \frac{\lambda}{N} d_{f_i} \left( \frac{\lambda}{N} \right)^\frac{1}{p} : \lambda>0 \right\} \\
        &= N \sum_{i=1}^N \|f_i\|_{p,\infty}
      \end{align*}

    \end{proof}

  \item (Exercise 1.1.12 in Grafakos \textit{Normability of weak-$L^p$ for $p>1$.}) Let $(X,\mu)$ be a $\sigma$-finite measure space and let $p>1$.
    Define
    \[ | \| f \| |_{p,\infty} = \sup_{0 < \mu(E) < \infty} \mu(E)^{\frac{1}{p} - 1} \int_{E}^{}|f| d \mu .\]

    \begin{enumerate}
      \item Prove that
        \[ | \| f \| |_{p,\infty} \leq \frac{p}{p-1} \|f\|_{p,\infty} ,\]
        where $\|f\|_{p,\infty}$ is the usual norm on $L^{p,\infty}$.

      \item Prove that
        \[ \|f\|_{p,\infty} \leq | \| f \| |_{p,\infty} .\]

      \item Prove that $| \| f \| |_{p,\infty}$ is a norm and deduce that $L^{p,\infty}$ is normable.

    \end{enumerate}

    \begin{proof}

      \begin{enumerate}
        \item
          Let $E$ be a set with $0<\mu(E)<\infty$. We have
          \begin{align*}
            \int_{E}^{} |f(x)| dx &= \int_{0}^{\infty} \mu( E \cap \{ |f| > \lambda \} ) d\lambda \\
            &\leq \int_{0}^{\infty} \min \{ \mu(E), d_f(\lambda) \} d\lambda \\
            &\leq \int_{0}^{\infty} \min \{ \mu(E), \lambda^{-p} \|f\|_{p,\infty}^p \} d\lambda
          \end{align*}

          Now, we are able to split our integral at any value $\alpha$ and use $\mu(E)$ with the finite interval and $\lambda^{-p} \|f\|_{p,\infty}^p$
          on the infinite interval. By optimizing the resulting inequality, we find the best splitting happens at $\alpha =
          \mu(E)^{-\frac{1}{p}} \|f\|_{p,\infty}$. This gives us

          \begin{align*}
            \int_{E}^{} |f(x)| dx &\leq \int_{0}^{\mu(E)^{-\frac{1}{p}} \|f\|_{p,\infty}} \mu(E) d\lambda + \int_{\mu(E)^{-\frac{1}{p}} \|f\|_{p,\infty}}^{\infty}
            \lambda^{-p} \|f\|_{p,\infty}^p d\lambda \\
            &= \mu(E)^{1-\frac{1}{p}} \|f\|_{p,\infty} \mu^{1-\frac{1}{p}} \frac{1}{p-1} \|f\|_{p,\infty} \\
            &= \mu(E)^{1-\frac{1}{p}} \frac{p}{p-1} \|f\|_{p,\infty}
          \end{align*}

          Taking the supremum over all such $E$, we get
          \[ | \|f\| |_{p,\infty} \leq \frac{p}{p-1} \| f \|_{p,\infty} .\]

        \item
          Take $f \in L^{p,\infty}$. Let $\lambda>0$. Define $E_\lambda = \{ |f| > \lambda \}$. Because $f \in L^{p,\infty}$, $\mu(E_\lambda) <
          \infty$.

          Then
          \begin{align*}
            \lambda d_f(\lambda)^{\frac{1}{p}} &= \lambda \mu(E_\lambda)^{\frac{1}{p}} \\
            &\leq \mu(E_\lambda)^{\frac{1}{p} - 1} \int_{E_\lambda}^{} |f(x)| dx \quad \parbox{5cm}{by Chebyshev's Inequality}
            &\leq | \|f\| |_{p,\infty}
          \end{align*}

          Taking the supremum over all $\lambda$, we get
          \[ \|f\|_{p,\infty} \leq | \|f\| |_{p,\infty} .\]

        \item
          Let $c \in \R$ and $f \in L^{p,\infty}$. Then
          \begin{align*}
            | \| cf \| |_{p,\infty} &= \sup_{0 < \mu(E) < \infty} \mu(E)^{\frac{1}{p}-1} \int_{E}^{} |cf| d\mu \\
            &= c \sup_{0 < \mu(E) < \infty} \mu(E)^{\frac{1}{p} - 1} \int_{E}^{} |f| d\mu \\
            &= c | \| f \| |_{p,\infty}
          \end{align*}

          Let $f,g \in L^{p,\infty}$. Then
          \begin{align*}
            | \| f+g \| |_{p,\infty} &= \sup_{0 < \mu(E) < \infty} \mu(E)^{\frac{1}{p} - 1} \int_{E}^{} |f+g| d\mu \\
            &\leq \sup_{0 < \mu(E) < \infty} \mu(E)^{\frac{1}{p} - 1} \left( \int_{E}^{} |f| d\mu + \int_{E}^{} |g| d\mu \right) \\
            &\leq \sup_{0 < \mu(E_1) < \infty} \mu(E_1)^{\frac{1}{p} - 1} \int_{E_1}^{} |f| d\mu + \sup_{0 < \mu(E_2) < \infty}
            \mu(E_2)^{\frac{1}{p} - 1} \int_{E_2}^{} |g| d\mu \\
            &= | \|f\| |_{p,\infty} + | \|g\| |_{p,\infty}
          \end{align*}

          Let $f \in L^{p,\infty}$ with $f \neq 0 \ a.e.$ Then there is some set $U$ with $0 < \mu(U) < \infty$ such that $|f|>0$ on $U$. Then
          \begin{align*}
            | \| f \| |_{p,\infty} &= \sup_{0 < \mu(E) < \infty} \mu(E)^{\frac{1}{p}-1} \int_{E}^{} |f| d\mu \\
            &\geq \mu(U)^{\frac{1}{p}-1} \int_{U}^{} |f| d\mu \\
            &>0
          \end{align*}

          Therefore, $| \| \cdot \| |_{p,\infty}$ is a norm on $L^{p,\infty}$ and $L^{p,\infty}$ is normable.
      \end{enumerate}

    \end{proof}

  \item (Exercise 2.1.13 in Grafakos) Observe that in the proof of the weak-(1,1) boundedness of the Hardy-Littlewood maximal function we have
    actually obtained the inequality
    \[ \lambda \cdot \mu( \{ M(f) > \lambda \} )^{\frac{1}{p}} \leq 3^n \mu( \{M(f) > \lambda \})^{\frac{1}{p} - 1} \int_{\{M(f)>\lambda\}}^{} |f|
    d\mu \]
    for $\lambda > 0$ and $f$ locally integrable. Use this fact and the previous exercise to deduce that $M$ maps $L^{p,\infty}$ into itself.

    \begin{proof}

      Let $f \in L^{p,\infty}$ for $p>1$. Then by definition, we have
      \begin{align*}
        \lambda \cdot \mu( \{M(f) > \lambda \})^{\frac{1}{p}} &\leq 3^n \mu( \{ M(f) > \lambda \} )^{\frac{1}{p}-1} \int_{\{ M(f) > \lambda \}}^{} |f|
        d\mu \\
        &\leq 3^n \sup_{0 < \mu(E) < \infty} \mu(E)^{\frac{1}{p}-1} \int_{E}^{} |f| d\mu \\
        &\leq 3^n \frac{p}{p-1} \|f\|_{p,\infty} \quad \parbox{5cm}{by Problem 3}
      \end{align*}

      Taking the supremum over all $\lambda$, we have
      \[ \| M(f) \|_{p,\infty} \leq 3^n \frac{p}{p-1} \|f\|_{p,\infty} \]
      and $M$ maps $L^{p,\infty}$ to itself.

    \end{proof}
\end{enumerate}

\end{document}



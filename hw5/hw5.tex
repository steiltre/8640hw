%        File: hw5.tex
%     Created: Fri Oct 28 11:00 AM 2016 C
% Last Change: Fri Oct 28 11:00 AM 2016 C
%

\documentclass[a4paper]{article}

\title{Math 8640 Homework 5 }
\date{11/4/16}
\author{Trevor Steil}

\usepackage{amsmath}
\usepackage{amsthm}
\usepackage{amssymb}
\usepackage{esint}
\usepackage{algorithm}
\usepackage{algorithmicx}
\usepackage{bbm}

\newtheorem{theorem}{Theorem}[section]
\newtheorem{corollary}{Corollary}[section]
\newtheorem{proposition}{Proposition}[section]
\newtheorem{lemma}{Lemma}[section]
\newtheorem*{claim}{Claim}
\newtheorem*{problem}{Problem}
%\newtheorem*{lemma}{Lemma}
\newtheorem{definition}{Definition}[section]

\newcommand{\R}{\mathbb{R}}
\newcommand{\N}{\mathbb{N}}
\newcommand{\C}{\mathbb{C}}
\newcommand{\Z}{\mathbb{Z}}
\newcommand{\Q}{\mathbb{Q}}
\newcommand{\supp}[1]{\mathop{\mathrm{supp}}\left(#1\right)}
\newcommand{\lip}[1]{\mathop{\mathrm{Lip}}\left(#1\right)}
\newcommand{\curl}{\mathrm{curl}}
\newcommand{\la}{\left \langle}
\newcommand{\ra}{\right \rangle}
\renewcommand{\vec}[1]{\mathbf{#1}}

\newenvironment{solution}[1][]{\emph{Solution #1}}

\algnewcommand{\Or}{\textbf{ or }}
\algnewcommand{\And}{\textbf{ or }}

\begin{document}
\maketitle

\begin{enumerate}
\item
  \begin{problem}
    (cf. Exercise 5.1.1 in Grafakos.)
    \begin{enumerate}
    \item Prove that there is some absolute constant $C>0$ ($C=4$ would work) such that  for any $0<a<b<\infty$
    $$ \bigg| \int\limits_a^b \frac{\sin x}{x} \, dx \bigg| \le C.$$
    \item Prove that the value of the {\it{Dirichlet integral}} is $\pi/2$, i.e. $$  \int\limits_0^\infty \frac{\sin x}{x} \, dx = \frac{\pi}2.$$
    Show that this integral is defined in the sense of improper Riemann integral, but is not defined in the sense of Lebesgue integration on $(0,\infty)$, in particular, $\frac{\sin{x}}{x}$ it is not absolutely integrable on $(0,\infty)$.
    Deduce that  for $b\in \mathbb R$, $b\neq 0$ $$  \int\limits_0^\infty \frac{\sin (bx)}{x} \, dx = \frac{\pi}2 \textup{sgn } (b).$$
    \end{enumerate}
  \end{problem}

  \begin{proof}

    \begin{enumerate}
      \item
        Let
        \[ a_n = \int_{n \pi}^{(n+1) \pi} \frac{\sin x}{x} dx \]
        for $n \in \N$. Then
        \[ \int_{0}^{\infty} \frac{\sin x}{x} dx = \sum_{n=0}^\infty a_n .\]
        Notice that this is an alternating series. By the periodicity of $\sin x$, we have $|a_n| > |a_{n+1}|$, and because $|\sin x| \leq 1$, $\lim_{n
        \to \infty} a_n = 0$. By the alternating series test, we know $\left| \int_{0}^{\infty} \frac{\sin x}{x} dx \right| = D < \infty$.

        For $0<a<b<\infty$, we can find $i,j \in \N$ such that $i \leq a < i+1$ and $j \leq b < j+1$. By choosing $k = j$ or $k=j+1$ based on the
        relative signs of $\sin a$ and $\sin b$, we have
        \[ \left| \int_{a}^{b} \frac{\sin x}{x} dx \right| \leq \left| \sum_{n=i}^k a_n \right| .\]
        Because the series is alternating and the terms are approaching zero,
        \begin{align*}
          \left| \sum_{n=i}^k a_n \right| &= \left| D - \sum_{n=0}^{i-1} a_n - \sum_{n=k+1}^\infty a_n \right| \\
          &\leq D + 2 a_0
        \end{align*}

        Therefore,
        \[ \left| \int_{a}^{b} \frac{\sin x}{x} dx \right| \leq C \]
        for $C = D + 2 a_0$.

      \item
        We will prove this by using Fubini's Theorem on the strip $(0,a) \times (0, \infty)$ and then taking $a \to \infty$ with the integral
        \[ \int_{0}^{a} \int_{0}^{\infty} e^{-st} \sin t ds dt .\]

        First,
        \begin{align*}
          \int_{0}^{a} \int_{0}^{\infty} \left| e^{-st} \sin t \right| ds dt &= \int_{0}^{a} \left| \frac{\sin t}{t} \right| dt \\
          &= \int_{0}^{1} \left| \frac{\sin t}{t} \right| dt + \int_{1}^{a} \left| \frac{\sin t}{t} \right| dt \\
          &\leq 1 + \ln a < \infty
        \end{align*}

        Therefore, we can apply Fubini's Theorem. We have
        \begin{align*}
          \int_{0}^{a} \int_{0}^{\infty} e^{-st} \sin t ds dt &= \int_{0}^{a} \frac{\sin t}{t} dt.
        \end{align*}

        We also have
        \begin{align*}
          \int_{0}^{\infty} \int_{0}^{a} e^{-st} \sin t dt ds &= \int_{0}^{\infty} -e^{-st} \frac{s \sin t + \cos t}{s^2+1} \bigg|_{0}^a ds \\
          &= \int_{0}^{\infty} \frac{1}{s^2 + 1} ds - \int_{0}^{\infty} e^{-as} \frac{s \sin a + \cos a}{s^2 + 1} ds \\
          &= \tan^{-1} s \big|_0^\infty - \int_{0}^{\infty} e^{-as} \frac{s \sin a + \cos a}{s^2 + 1} ds \\
          &= \frac{\pi}{2} - \int_{0}^{\infty} e^{as} \frac{s \sin a + \cos a}{s^2+1} ds.
        \end{align*}

        Taking $a \to \infty$ and using Fubini's Theorem, we have
        \begin{align*}
          \int_{0}^{\infty} \frac{\sin t}{t} dt &= \frac{\pi}{2} - \lim_{a \to \infty} \int_{0}^{\infty} e^{-as} \frac{s \sin a + \cos a}{s^2 + 1} ds
          \\
          &= \frac{\pi}{2} \quad \parbox{6cm}{by Lebesgue Dominated Convergence}
        \end{align*}

        Now we will show $\frac{\sin t}{t}$ is not absolutely integrable. Because $\sin t$ is continuous, for any $\varepsilon > 0$, there is a
        $\delta>0$ such that for any $n$, $\left| \sin \left( \frac{\pi}{2} + \pi n + x \right) \right| > 1 - \varepsilon$ for $\left| x - \frac{\pi}{2} + \pi n
        \right| < \delta$.

        Then we have
        \begin{align*}
          \int_{0}^{\infty} \left| \frac{\sin t}{t} \right| dt &\geq \sum_{n=0}^\infty \int_{\left| x - \frac{\pi}{2} + \pi n \right| < \delta}^{}
          \left| \frac{\sin t}{t} \right| dt \\
          &> \sum_{n=1}^\infty \int_{\left| x - \frac{\pi}{2} + n \pi \right| < \delta}^{} \frac{1-\varepsilon}{\frac{\pi}{2} + n \pi - \delta} dt \\
          &= (1-\varepsilon) \delta \sum_{n=1}^\infty \frac{1}{\frac{\pi}{2} + n \pi - \delta} \\
          &> (1-\varepsilon)\delta \sum_{n=1}^\infty \frac{1}{n \pi} \quad \parbox{5cm}{for $\delta < \frac{\pi}{2}$} \\
          &=\infty
        \end{align*}

        Therefore, $\frac{\sin t}{t}$ is not absolutely integrable.

        We can deduce
        \[ \int_{0}^{\infty} \frac{\sin (bx)}{x}dx = \frac{\pi}{2} \textup{ sgn} (b) \]
        through a simple change of variables.

        For $b \geq 0$, use $y = bx$ to get
        \[ \int_{0}^{\infty} \frac{\sin(bx)}{x} dx = \int_{0}^{\infty} \frac{\sin y}{y} dy = \frac{\pi}{2} .\]

        For $b < 0$, use $y = -bx$ to get
        \begin{align*}
          \int_{0}^{\infty} \frac{\sin(bx)}{x} dx &= \int_{0}^{\infty} \frac{\sin (-y)}{y}dy \\
          &= - \int_{0}^{\infty} \frac{\sin y}{y} dy \\
          &= - \frac{\pi}{2}
        \end{align*}

        Therefore, we have
        \[ \int_{0}^{\infty} \frac{\sin(bx)}{x}dx = \frac{\pi}{2} \textup{ sgn}(b) .\]

    \end{enumerate}

  \end{proof}
\item
  \begin{problem}
    (Exercise 5.1.3 in Grafakos.) Use the Fourier multiplier representation of the Hilbert transform to define $H(f)$ as an element of $\mathcal S' (\mathbb R)$ for bounded functions $f$ whose  (distributional) Fourier transform vanishes in the neighborhood of the origin. Show that with this definition $$ H (\cos x) = \sin x.$$
  \end{problem}

  \begin{proof}

    Let $g \in \mathcal{S}$. Then
    \begin{align*}
      \la H(\cos x), g \ra &= \la \cos x, H^{\ast} g \ra \\
      &= \la (\cos x)^{\check{}}, (H^{\ast} g)^{\widehat{}} \ \ra \\
      &= \la (\cos x)^{\check{}}(\xi), i \textup{ sgn} (\xi) \widehat{g}(\xi) \ra \quad \parbox{6cm}{because $H^\ast = -H$ and using the Hilbert
      transform's Fourier multiplier} \\
      &= \int_{-\infty}^{\infty} \frac{1}{2} \left( \delta(\xi - 1) + \delta(\xi + 1) \right) i \textup{ sgn}(\xi) \widehat{g}(\xi) d\xi \\
      &= \int_{-\infty}^{\infty} \frac{i}{2} \left( \delta(\xi - 1) - \delta(\xi + 1) \right) \widehat{g}(\xi) d\xi \\
      &= \la (\sin x)^{\check{}}, \widehat{g} \ra \\
      &= \la \sin x, g \ra
    \end{align*}

    Therefore, $H(\cos x) = \sin x$.

  \end{proof}

\item
  \begin{problem}
    (Exercise 5.1.4 in Grafakos.)  Let $E \subset \mathbb R$ be a measurable set with finite measure. Show that the distribution function of $H ({\bf{1}}_E)$ satisfies $$ d_{H ({\bf{1}}_E)} (\lambda)  = \frac{ 4|E| }{ e^{\pi \lambda}  - e^{-\pi \lambda}}$$ for all $\lambda >0$.

    Conclude that for all $f$ of the form $f = {\bf{1}}_E$  the Hilbert transform satisfies a weak $(1,1)$ inequality, i.e. $\| H (f) \|_{1,\infty} \le C \| f \|_1$.

    Remark: this is called ``restricted weak type" and it suffices for real interpolation (See Stein or Grafakos).

    Hint: First consider $E$ to be an interval (we computed $H( {\bf{1}}_{(a,b)})$ in class), then consider a finite union of intervals, and then approximate an arbitrary measurable set by unions of intervals. For a more extensive and detailed hint, see Grafakos.

    Alternatively: prove that $H$ has  restricted weak type $(1,1)$ in any other way (avoiding the Calder\'{o}n--Zygmund decomposition).

  \end{problem}

  \begin{solution}

    First, assume $E = \bigcup_{i=1}^n (a_i, b_i)$ is a union of disjoint intervals. Then we have
    \[ H( \mathbbm{1}_E )(x) = \frac{1}{\pi} \log \left( \prod_{i=1}^n \left| \frac{x-b_i}{x-a_i} \right| \right) .\]
    We see that
    \[ \lim_{x \to b_i} H( \mathbbm{1}_E )(x) = -\infty, \]
    and
    \[ \lim_{x \to a_i} H( \mathbbm{1}_E )(x) = \infty. \]
    Also, between these singularities, $H( \mathbbm{1}_E )(x)$ is either monotonically increasing or monotonically decreasing. Thus for any $\lambda >
    0$ and any $1 \leq i \leq n$, there is exactly one $x \in (a_i, b_i)$ such that $H( \mathbbm{1}_E )(x) = \lambda$. Call these roots $r_j$.
    Similarly, there is exactly one $x \in (b_j, a_{j+1})$ such that $H( \mathbbm{1}_E )(x) = \lambda$. Call these roots $\rho_j$.

    By the monotonicity of $H( \mathbbm{1}_E )(x)$, we have
    \[ \mu \left( \left\{ H( \mathbbm{1}_E ) > \lambda \right\} \right) = \sum_{j=1}^n r_j - \sum_{j=1}^n \rho_j .\]

    Let's look more closely at solutions of $H( \mathbbm{1}_E )(x) = \lambda$. First, we can rewrite this equation as
    \begin{equation} \label{eqn:roots}
      \prod_{i=1}^n \left| \frac{x-b_i}{x-a_i} \right| = e^{\pi \lambda} .
    \end{equation}
    For solutions in $(a_j, b_j)$, we know
    \[ \left| \frac{x-b_i}{x-a_i} \right| =
      \begin{cases}
        \frac{x-b_i}{x-a_i} &\text{if } i \neq j \\
        - \frac{x-b_i}{x-a_i} &\text{if } i = j
      \end{cases}
    \]

    In this case, \eqref{eqn:roots} can be rewritten as
    \[ \prod_{i=1}^n (x - b_i) = -e^{\pi \lambda} \prod_{i=1}^n (x-a_i) .\]

    For a polynomial, $c_n x^n + c_{n-1} x^{n-1} + \dots + c_0$, the sum of the roots can be expressed as $-\frac{c_{n-1}}{c_n}$. In this case, the
    particular roots were defined to be $r_j$, so we get
    \[ \sum_{i=1}^n r_i = \frac{\sum_{i=1}^n \left( b_i + a_i e^{\pi \lambda} \right)}{1+e^{\pi \lambda}} .\]

    In a similar manner, we can find
    \[ \sum_{i=1}^n \rho_i = \frac{\sum_{i=1}^n \left(b_i - a_i e^{\pi \lambda} \right)}{1 - e^{\pi \lambda}} .\]

    This gives us
    \begin{align*}
      \mu \left( \left\{ H( \mathbbm{1}_E ) > \lambda \right\} \right) &= \sum_{j=1}^n r_j - \sum_{j=1}^n \rho_j \\
      &= \frac{\sum_{j=1}^n \left( b_j + a_j e^{\pi \lambda} \right)}{1+e^{\pi \lambda}} -
      \frac{\sum_{j=1}^n \left( b_j - a_j e^{\pi \lambda} \right)}{1 - e^{\pi \lambda}} \\
      &= \frac{\sum_{j=1}^n \left[ \left(b_j + a_j e^{\pi \lambda} \right) \left( 1 - e^{\pi \lambda} \right) - \left(b_j - a_j e^{\pi \lambda}
      \right) \left( 1+e^{\pi \lambda} \right) \right]}{1 - e^{2 \pi \lambda}} \\
      &= \frac{\sum_{j=1}^n 2 \left( a_j - b_j \right) e^{\pi \lambda}}{1 - e^{2 \pi \lambda}} \\
      &= \frac{\sum_{j=1}^n 2 \left( b_j - a_j \right)}{e^{\pi \lambda} - e^{- \pi \lambda}} \\
      &= \frac{2 |E|}{e^{\pi \lambda} - e^{-\pi \lambda}}
    \end{align*}

    Looking at solutions to $H( \mathbbm{1}_E )(x) = -\lambda$, we get the same result. Therefore, we have that
    \[ d_{H( \mathbbm{1}_E )} (\lambda) = \frac{4 |E|}{e^{\pi \lambda} - e^{-\pi \lambda}} \]
    for $E$ being a finite union of disjoint intervals.

    Now take $E$ to be any measurable subset of $\R$ with finite measure. Then $\mathbbm{1}_E$ is a measurable function, so we can take subsets $E_n$
    of $\R$ which are finite unions of intervals such that $\mathbbm{1}_{E_n} \to \mathbbm{1}_{E}$ in $L^1$. Then $H( \mathbbm{1}_{E_n} ) \to H(
    \mathbbm{1}_{E})$ because $H: L^1 \to L^{1,\infty}$. But this means $d_{H( \mathbbm{1}_{E_n} )} \to d_{H( \mathbbm{1}_E )}$ and the result holds
    for measurable $E$.

  \end{solution}

\item
  \begin{problem}
    (Exercise 5.1.10 in Grafakos.)  Prove the distributional (i.e. in the sense of $\mathcal S' (\mathbb R^n)$) identity $$ \partial_j |x|^{-n+1}  = (1-n) \textup{ p.v. } \frac{x_j}{|x|^{n+1}}.$$ Use it to give a proof of the Fourier multiplier representation of the Riesz transform.
  \end{problem}

  \begin{proof}

    Take any $\varphi \in \mathcal{S}$ and let $\psi \in \mathcal{C}^\infty(\R)$ be even with $\psi(x) = 0$ for $0 \leq |x| \leq \frac{1}{2}$, and $\psi(x) = 1$ for $|x| \geq 1$. We will
    use the standard dilation given by $\psi_\varepsilon(x) = \psi \left( \frac{x}{\varepsilon} \right)$. We will abuse notation slightly and use
    $\psi(x)$ to mean $\psi(|x|)$. Using this, we can write
    \begin{align*}
      \la \partial_j |x|^{-n+1} , \varphi \ra &= \la \psi_\varepsilon \partial_j |x|^{-n+1}, \varphi \ra + \la (1-\psi_\varepsilon) \partial_j |x|^{-n+1}, \varphi \ra \\
      &=  \la \partial_j |x|^{-n+1}, \psi_\varepsilon \varphi \ra - \la |x|^{-n+1}, \partial_j \left[ (1 - \psi_\varepsilon) \varphi \right] \ra \\
    \end{align*}

    In the first term, we see $\psi_\varepsilon \varphi$ is supported away from 0, so we are able to take the derivative of $|x|^{-n+1}$ to get
    \begin{align*}
      \la \partial_j |x|^{-n+1}, \psi_\varepsilon \varphi \ra &= \la (1-n) \frac{x_j}{|x|^{n+1}}, \psi_\varepsilon \varphi \ra \\
      &= \int_{|x|>\varepsilon}^{} (1-n) \frac{x_j}{|x|^{n+1}} \varphi dx + \int_{\frac{\varepsilon}{2} \leq |x| \leq \varepsilon}^{}
      \frac{x_j}{|x|^{n+1}} \psi_\varepsilon \varphi dx
    \end{align*}

    As we take $\varepsilon \to 0$, we will have the second term go to 0, leaving
    \[ \la \partial_j |x|^{-n+1}, \psi_\varepsilon \varphi \ra = \la (1-n) \textup{ p.v. } \frac{x_j}{|x|^{n+1}}, \varphi \ra .\]

    For the second term, we have
    \begin{align*}
      - \la |x|^{-n+1}, \partial_j \left[ (1-\psi_\varepsilon) \varphi \right] \ra &= \la |x|^{-n+1}, \varphi \partial_j \psi_\varepsilon \ra - \la
      |x|^{-n+1}, (1 - \psi_\varepsilon) \partial_j \varphi \ra \\
    \end{align*}

    Because $|x|^{-n+1} \partial_j \psi_\varepsilon$ is odd, we can add a constant to our integral, giving
    \begin{align*}
      \int_{\R^n}^{} |x|^{-n+1} \partial_j \psi_\varepsilon(x) \varphi(x) dx &= \int_{\R^n}^{} |x|^{-n+1} \partial_j \psi \left( \frac{x}{\varepsilon}
      \right) \frac{\varphi(x) - \varphi(0)}{\varepsilon} dx
    \end{align*}
    Bounding this using $\varphi'$ and using the fact that $\partial_j \psi \left( \frac{x}{\varepsilon} \right) \to 0$, this term goes to zero.
    Because $(1 - \psi_\varepsilon) \to 0$, we get $\la |x|^{-n+1}, (1 - \psi_\varepsilon) \partial_j \varphi \ra \to 0$ as well. This leaves our
    principal value term as the only nonzero term, so
    \[ \la \partial_j |x|^{-n+1}, \varphi \ra = \la (1-n) \textup{ p.v. } \frac{x_j}{|x|^{n+1}}, \varphi \ra .\]

    Taking Fourier Transforms, we have
    \[ \la \mathcal{F} [ (1-n) \textup{ p.v. } \frac{x_j}{|x|^{n+1}} ], f \ra = \la \frac{\pi^{\frac{n+1}{2}}}{\Gamma \left( \frac{n+1}{2} \right)}
    (1-n) \mathcal{F} (W_j) , f \ra, \]
    and
    \begin{align*}
      \la \mathcal{F} [\partial_j |x|^{-n+1}], f \ra &= \la 2 \pi i \xi_j \mathcal{F}[ |x|^{-n+1}], f \ra \\
      &= \la 2 \pi i \frac{\pi^{\frac{n-1}{2}}}{\Gamma \left( \frac{n-1}{2} \right)} \frac{\xi_j}{|\xi|}, f \ra \quad \parbox{5cm}{by Theorem
      2.4.6 in Grafakos}
    \end{align*}

    Using properties of the Gamma function, we then have
    \[ \widehat{W}_j = -i \frac{\xi_j}{|\xi|} \]
    as distributions.

  \end{proof}

\item
  \begin{problem}
    (Exercise 5.1.11 in Grafakos). Prove that if $T$ is a linear operator, which is bounded on $L^2 (\mathbb R)$, commutes with translations and dilations, and anticommutes with reflections (i.e. $ T (\widetilde{f}\,) = - \widetilde{Tf}$, where $\widetilde{f} (x) = f (-x)$), then $T$ is a constant multiple of the Hilbert transform.
  \end{problem}

  \begin{proof}

    Because $T$ is bounded from $L^2$ to $L^2$ and commutes with translations, we know $\widehat{Tf} = m \widehat{f}$ for some $m \in L^\infty$.

    $T$ anticommutes with reflections, so
    \begin{align*}
      m \tilde{\widehat{f}} &= m \widehat{\tilde{f}} \\
      &= \widehat{T( \tilde{f} ) } \\
      &= - \widehat{ \tilde{T(f)} } \\
      &= - \tilde{ \widehat{T(f)} } \\
      &= - \tilde{m} \tilde{ \widehat{f} }
    \end{align*}

    Therefore, $m = - \tilde{m}$, that is, $m$ is an odd function.

    Let $f_a(x) = f(ax)$ for any $a>0$. $T$ commutes with dilations, so
    \begin{align*}
      \frac{1}{|a|} m \left( \widehat{f} \right)_{\frac{1}{a}} &= m (f_a)^{\widehat{}} \\
      &= \widehat{T(f_a)} \\
      &= \left( \widehat{Tf} \right)_a \\
      &= \frac{1}{|a|} \left( \widehat{Tf} \right)_{\frac{1}{a}} \\
      &= \frac{1}{|a|} ( m \widehat{f} )_{\frac{1}{a}} \\
      &= \frac{1}{|a|} m_{\frac{1}{a}} ( \widehat{f} )_{\frac{1}{a}}
    \end{align*}

    Therefore, $m(\xi) = m(\frac{1}{a} \xi)$ for all $\xi$. Combining this with the fact that $m$ is odd, we know $m(\xi) = c \textup{ sgn} (\xi)$ for
    some $c>0$, and $T$ is a multiple of the Hilbert transform.

  \end{proof}

\item
  \begin{problem}
    (Exercise 5.3.5 in Grafakos). Assume that a linear operator  $ T$ maps $L^p (\mathbb R^n)$ to itself for some $p >1$ and assume that it satisfies the following property: if $f$ is supported in a cube $Q$, then $Tf$ is supported in a fixed multiple of $Q$, i.e. a cube $Q^* = c Q$ with the same center as $Q$ and $c$-times the side length (for some fixed $c>0$).

    Prove that $T$ is of weak type $(1,1)$, i.e. $T$ maps $L^1$ into $L^{1,\infty}$.

    Hint: use the Calder\'{o}n--Zygmund decomposition.
  \end{problem}

  \begin{proof}

    Take $f \in L^1$. Using the Calder\'{o}n--Zygmund decomposition, we can write $f = g+b$ where
    \begin{enumerate}
      \item $\|g\|_1 \leq \|f\|_1$
      \item $ \|g\|_\infty \leq 2^{n} \alpha$
      \item $b = \sum_j b_j$ with the $b_j$ supported on disjoint cubes $Q_j$
      \item $\int_{Q_j}^{}b_j dx = 0$
      \item $\| b_j \|_1 \leq 2^{n+1}\alpha |Q_j|$
      \item $ \sum_j |Q_j| \leq \frac{1}{\alpha} \| f \|_1$
    \end{enumerate}
    where $\alpha > 0$ will be determined later.

    We have
    \[ \mu \left( \{ |Tf| > \lambda \} \right) \leq \mu \left( \left\{ |Tg| > \frac{\lambda}{2} \right\} \right) + \mu \left( \left\{ |Tb| >
      \frac{\lambda}{2} \right\} \right) .\]

    We will look at each of these terms individually. First,
    \begin{align*}
      \mu\left( \left\{ |Tg| > \frac{\lambda}{2} \right\} \right) &\leq \frac{2^p}{\lambda^p} \|Tg\|_p^p \quad \parbox{5cm}{by Chebyshev's inequality}
      \\
      &\leq \frac{2^p \|T\|_{p \to p}^p}{\lambda^p} \|g\|_p^p \quad \parbox{5cm}{by $L^p$ boundedness of $T$} \\
      &\leq \frac{2^p \|T\|_{p \to p}^p}{\lambda^p} 2^{n(p-1)} \alpha^{p-1} \|g\|_1 \quad \parbox{5cm}{by (b)} \\
      &\leq \frac{2^p \|T\|_{p \to p}^p}{\lambda^p} 2^{n(p-1)} \alpha^{p-1} \|f\|_1 \quad \parbox{5cm}{by (a)}
    \end{align*}

    By choosing $\alpha = \lambda$, we get
    \[ \mu \left( \left\{ |Tg| > \frac{\lambda}{2} \right\} \right) \leq \frac{\|T\|_{p \to p}^p}{\lambda} 2^{n(p-1) + p} \|f\|_1 .\]

    For the second term, we have $Tb_j$ is supported on $c Q_j$. The $Q_j$ are disjoint, so there is a constant $N$ depending on $C$ and $n$ such that
    at most $N$ of the $c Q_j$ intersect at any point. Therefore, ${ \mu \left( \left\{ \sum_j |Tb_j| > \frac{\lambda}{2} \right \} \right) \leq \sum_j
    \mu \left( \left\{ |Tb_j| > \frac{\lambda}{2N} \right\} \right) }$. Using this fact, we have
    \begin{align*}
      \mu \left( \left\{ \left| T \sum_j b_j \right| > \frac{\lambda}{2} \right\} \right) &\leq \mu \left( \left\{ \sum_j | T b_j | >
        \frac{\lambda}{2} \right\} \right) \quad \parbox{7cm}{because the $b_j$ have disjoint support} \\
      &\leq \sum_j \mu \left( \left\{ |T b_j | > \frac{\lambda}{2N} \right\} \right) \\
      &\leq \sum_j c |Q_j| \quad \parbox{5cm}{because $\supp{Tb_j} \subset c Q_j$} \\
      &\leq \frac{c}{\lambda} \|f\|_1 \quad \parbox{5cm}{by (f)}
    \end{align*}

    Combining our terms, we have
    \[ \mu \left( \left\{ |Tf| > \lambda \right\} \right) \leq \frac{C}{\lambda} \|f\|_1 \]
    for some constant $C$.

  \end{proof}

\end{enumerate}


\end{document}



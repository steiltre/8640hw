%        File: hw1.tex
%     Created: Sat Sep 10 06:00 PM 2016 C
% Last Change: Sat Sep 10 06:00 PM 2016 C
%

\documentclass[a4paper]{article}

\title{Homework 1 }
\date{9/19/16}
\author{Trevor Steil}

\usepackage{amsmath}
\usepackage{amsthm}
\usepackage{amssymb}
\usepackage{esint}
\usepackage{enumerate}

\newtheorem{theorem}{Theorem}[section]
\newtheorem{corollary}{Corollary}[section]
\newtheorem{proposition}{Proposition}[section]
\newtheorem{lemma}{Lemma}[section]
\newtheorem*{claim}{Claim}
\newtheorem*{problem}{Problem}
%\newtheorem*{lemma}{Lemma}
\newtheorem{definition}{Definition}[section]

\newcommand{\R}{\mathbb{R}}
\newcommand{\N}{\mathbb{N}}
\newcommand{\C}{\mathbb{C}}
\newcommand{\Z}{\mathbb{Z}}
\newcommand{\Q}{\mathbb{Q}}
\newcommand{\supp}[1]{\mathop{\mathrm{supp}}\left(#1\right)}
\newcommand{\lip}[1]{\mathop{\mathrm{Lip}}\left(#1\right)}
\newcommand{\curl}{\mathrm{curl}}
\newcommand{\la}{\left \langle}
\newcommand{\ra}{\right \rangle}
\renewcommand{\vec}[1]{\mathbf{#1}}

\newenvironment{solution}{\emph{Solution.}}

\begin{document}
\maketitle

\begin{enumerate}

\item Let $\{ f_k \}_{k\in \mathbb N}$ be an orthonormal system in $L^2 [0,1]$. Assume that for each $k\in \mathbb N$, the functions $f_k$ are continuously differentiable on $[0,1]$. Show that the family $\{ f'_k \}$ cannot be uniformly bounded.

  \begin{proof}

    We will prove the statement by contradiction. Assume $\{ f_k' \}$ is uniformly bounded. That is, assume there is some $C \geq 0$ such that
    $|f'_k(x)| \leq C$ for all $k \in \N$ and all $x \in [0,1]$.

    First, we will show the $\{ f_k \}$ are uniformly bounded. If not, for some $x_0 \in [0,1]$, we could find a subsequence $\{ f_{k_n} \}$ such that
    $| f_{k_n}(x_0) | > n$. Let $\delta > 0$. By the Mean Value Theorem and our uniform bound on $f'_{k_n}$, we have $f_{k_n}(x) \geq n - C \delta$
    for all $x \in [x - \delta, x + \delta]$ (in the case that $x=0$ or $x=1$, use an appropriate interval with 0 or 1 as an endpoint and the same
    argument holds). Therefore,
    \[ \int_{x-\delta}^{x+\delta} | f(x) |^2 dx \geq (n - C \delta)^2 * 2\delta \to \infty \text{ as } n \to \infty .\]
    This contradicts $\|f_k\| = 1$ for all $k$, therefore, $\{ f_k \}$ is uniformly bounded.

    Next, we will show the existence of a subsequence $\{ f_{k_n} \}$ and an $x \in [0,1]$ such that $f_{k_n}(x) \to y$ as $n \to \infty$ with $y \neq
    0$.

    Assume no such sequence exists. Take $x = 0$. $\{ f_k(0) \}$ is bounded by the uniform bound we already found, so $\{ f_{k_n}(0) \}$ has a
    convergent subsequence, $\{ f_{k_n} (0) \}$. By our assumption, we must have $f_{k_n}(0) \to 0$ as $n \to \infty$ (otherwise we have the sequence
    we assumed didn't exist).

    By repeating this argument countably many times, we can get a subsequence $\{ f_{k_n} \}$ such that $f_{k_n}(q) \to 0$ as $n \to \infty$ for all
    $q \in \Q$.

    Let $\varepsilon >0$. For any $x,y \in [0,1]$ with $|x-y| < \frac{\varepsilon}{C}$ where $C$ is the uniform bound on our derivatives, we have
    \[ | f_{k_n}(x) - f_{k_n}(y) | \leq \varepsilon .\]

    Thus, $\{f_{k_n}\}$ is a uniformly bounded and equicontinuous sequence of functions. Therefore, by Arzela-Ascoli, there is a further subsequence
    (which we will continue denoting by $\{ f_{k_n} \}$ that converges uniformly.

    Because $\{ f_{k_n} \}$ is a uniformly convergent sequence of continuous functions, it converges to a continuous function $f(x)$. As already
    stated $f(q) = 0$ for all $q \in \Q$, therefore $\|f\|_{L^2} = 0$. $f_{k_n} \to f$ uniformly implies $f_{k_n} \to f$ in $L^2$. Therefore,
    $\|f_{k_n}\|_{L^2} \to 0$, which contradicts $\|f_k\|_{L^2} = 1$ for all $k$. Therefore, there must be a point $x_1 \in [0,1]$ such that
    $f_{k_n}(x_1) \to y$ for some subsequence $\{ f_{k_n} \}$ and some $y \neq 0$.

    Then for $x \in [x_1 - \frac{y}{2C}, x_1 + \frac{y}{2C}]$, $|f(x)| \geq \frac{y}{2}.$ Then
    \begin{align*}
      \langle f, \chi_{[x_1 - \frac{y}{2C}, x_1 + \frac{y}{2C}]} \rangle_{L^2} &= \int_{x_1 - \frac{y}{2C}}^{x_1 + \frac{y}{2C}} |f(x)|^2 dx \\
      &\geq
    \end{align*}<++>

  \end{proof}

\item Let $D_N (x)$ be the Dirichlet kernel, i.e. $$ D_N (x) = \sum_{n=-N}^N e^{2\pi i nx}. $$

\begin{enumerate}[(a)]
\item Show that $$ D_N (x)  = \frac{\sin \big( (2N+1) \pi x\big)}{ \sin (\pi x)}. $$
\item Show that for any $\delta > 0$ there exists a constant $c_\delta >0$, such that for any $N \in \mathbb N$ $$ \int\limits_{\delta \le |x| \le 1/2} |D_N (x) | dx \ge c_\delta,$$
i.e. property $(iii)$ in the definition of approximate identities fails.
\item Show that   $$ \| D_N \|_1 \approx \log N,$$ more precisely that there exist constants $c$, $C>0$, such that $ c\log N \le  \| D_N \|_1 \le C \log N$ holds for all $N$ large enough,  i.e. property $(ii)$ in the definition of approximate identities fails.

{\it{Remark:}}  A more precise estimate is $$ \frac{4}{\pi^2} \sum_{k=1}^N \frac1{k} \le \| D_N \|_1 \le 2 + \frac{\pi}{4} +   \frac{4}{\pi^2} \sum_{k=1}^N \frac1{k},$$ but I only ask you to show that $\| D_N \|_1$ grows logarithmically.
\end{enumerate}

\begin{proof}
  \begin{enumerate}
    \item
      We compute
      \begin{align*}
        D_N(x) &= \sum_{n=-N}^N e^{2 \pi i n x} \\
        &= e^{-2 \pi i N x} \sum_{n=0}^{2N} e^{2 \pi i n x} \\
        &= e^{-2 \pi N x} \frac{1 - e^{\pi i (2N+1)x}}{1 - e^{2 \pi i x}} \\
        &= \frac{e^{-\pi i (2N+1)x} - e^{\pi i (2N+1) x}}{ e^{-\pi i x} - e^{ \pi i x}} \\
        &= \frac{\sin \left( \pi (2N+1)x \right)}{\sin \left( \pi x \right)}
      \end{align*}

    \item

    \item
      \begin{align*}
        \| D_N \|_1 &= \left\| \sum_{n = -N}^N e^{2 \pi i n x} \right\|_1 \\
        &= \int_{0}^{1} \left| \sum_{n=-N}^N e^{2 \pi i n x} \right| dx \\
        &= \int_{0}^{1} \left| 1 + 2 \sum_{n = 0}^N \cos( 2 \pi n x) \right| dx \\
        &\leq 1 + 2 \sum_{n = 0}^N \int_{0}^{1} \left| \cos(2 \pi n x) \right| dx \\
        &= 1 + \sum_{n=0}^N \frac{1}{\pi n} \int_{0}^{2 \pi n x} | \cos y | dy \\
      \end{align*}<++>

      \begin{align*}
        \| D_N \|_1 &= \int_{0}^{1} \left| \frac{\sin \left( (2N+1) \pi x \right)}{\sin (\pi x)} \right| dx \\
        &= \int_{0}^{1} \frac{ | \sin \left( ( 2N+1 ) \pi x \right) |}{ \pi x - \frac{(\pi x)^3}{3!} + \dots} dx \\
        &\geq \int_{0}^{1} \frac{| \sin \left( ( 2N+1 ) \pi x \right) |}{\pi x} dx \\
        &= \sum_{k=0}^{2N} \int_{\frac{k}{2N+1}}^{\frac{k+1}{2N+1}} \frac{| \sin \left( ( 2N+1 ) \pi x \right |)}{\pi x} dx \\
        &\geq \sum_{k = 0}^{2N} \frac{2N+1}{\pi (k+1)} \int_{\frac{k}{2N+1}}^{\frac{k+1}{2N+1}} | \sin \left( (2N+1) \pi x \right) | dx \\
      \end{align*}

      Now we integrate
      \begin{align*}
        \int_{\frac{k}{2N+1}}^{\frac{k+1}{2N+1}} \sin \left( (2N+1) \pi x \right) dx &= \frac{1}{(2N+1)\pi} \int_{k \pi}^{ (k+1) \pi } \sin y dy \\
        &= \frac{-1}{(2N+1) \pi} \cos y \big|_{k\pi}^{(k+1)\pi} \\
        &= (-1)^k \frac{2}{(2N+1) \pi}
      \end{align*}

      Therefore,
      \[ \| D_N \|_1 \geq \frac{2}{\pi^2} \sum_{k=0}^{2N} \frac{1}{k+1} \geq \frac{2}{\pi^2} \log (2N) .\]

  \end{enumerate}

\end{proof}

\item (Grafakos \#3.1.3) Let $F_N (x)$ be the Fej\'{e}r  kernel, i.e. $$ F_N (x) = \frac1{N+1} \sum_{n=0}^N D_n (x). $$

\begin{enumerate}[(a)]
\item Show that $$ F_N (x)  = \frac{1}{N+1} \left( \frac{\sin \big( (N+1) \pi x\big)}{ \sin (\pi x)} \right)^2 , $$ in particular $F_N (x) \ge 0$.
\item Show that the family $\{ F_N \}$ is an {\it{approximate identity}} as $N\rightarrow \infty$.

\end{enumerate}

\begin{proof}
  \begin{enumerate}
    \item
      First, we compute
      \begin{align}
        \sum_{n=0}^N \sin \left( (2n+1) \pi x \right) \sin ( \pi x ) &= \sum_{n=0}^N \frac{1}{2} \big[ - \cos \left( 2 (n+1) \pi x \right) + \cos
        \left( 2 n \pi x \right) \big] &\text{by a product identity} \nonumber \\
        &= \frac{1}{2} \big[ - \cos \left( 2(N+1) \pi x \right) + 1 \big] &\text{because our sum telescopes} \nonumber \\
        &= \frac{1}{2} \big[ - 1 + 2 \sin^2 \left( (N+1) \pi x \right) + 1 \big] &\text{by the double-angle formula} \nonumber \\
        &= \sin^2 \left( (N+1) \pi x \right)
        \label{eqn:trig_prod}
      \end{align}

      Therefore,
      \begin{align*}
        F_N(x) &= \sum_{n=0}^N D_n(x) \\
        &= \sum_{n=0}^N \frac{\sin \left( (2n+1) \pi x \right)}{\sin (\pi x)} \\
        &= \sum_{n=0}^N \frac{\sin \left( (2n+1) \pi x \right) \sin ( \pi x )}{\sin^2 (\pi x)} \\
        &= \left( \frac{\sin \left( (N+1) \pi x \right)}{\sin (\pi x)} \right)^2 \quad \text{by \eqref{eqn:trig_prod}}
      \end{align*}

    \item
  \end{enumerate}<++>

\end{proof}

\item  (Grafakos \#3.1.4) Let $V_N (x)$ be the de la Val\'{e}e Poussin   kernel, i.e. $$ V_N (x) = 2 F_{2N+1} (x) - F_N (x). $$

\begin{enumerate}[(a)]
\item Show that the family $\{ V_N \}$ is an {\it{approximate identity}} as $N\rightarrow \infty$.
\item Find and plot the Fourier coefficients of $V_N$.
\end{enumerate}
\item  (Grafakos \#3.1.11; Theorem of Fej\'er and F.\~Riesz)  Let $\displaystyle{ P(x) = \sum_{k=-N}^N a_k e^{2\pi i k x} }$ be a trigonometric polynomial such that $ P(x) > 0$ for all $x\in \mathbb T$. Prove that there exists a polynomial  $Q(x)$ of the form  $\displaystyle{ Q(x) = \sum_{k=0}^N b_k e^{2\pi i k x} }$ such that $$P (x) = | Q(x) |^2 . $$

{\it{Hint:}}  Note that $N$ zeros of the complex polynomial   $\displaystyle{ R(z) = \sum_{k=-N}^N a_k z^{k+N} }$ lie inside the unit circle, while the other $N$ lie outside.

\end{enumerate}

\end{document}



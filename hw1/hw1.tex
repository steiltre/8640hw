\documentclass[12pt]{article}

\usepackage{amsmath,amstext,amssymb,enumerate}
\usepackage[colorlinks,citecolor=red,pagebackref,hypertexnames=false]{hyperref}
\usepackage[backrefs]{amsrefs}



\begin{document}


{\bf{Homework \#1}}


Due in class on Monday, Sep 19.


\begin{enumerate}

\item Let $\{ f_k \}_{k\in \mathbb N}$ be an orthonormal system in $L^2 [0,1]$. Assume that for each $k\in \mathbb N$, the functions $f_k$ are continuously differentiable on $[0,1]$. Show that the family $\{ f'_k \}$ cannot be uniformly bounded.

\item Let $D_N (x)$ be the Dirichlet kernel, i.e. $$ D_N (x) = \sum_{n=-N}^N e^{2\pi i nx}. $$

\begin{enumerate}[(a)]
\item Show that $$ D_N (x)  = \frac{\sin \big( (2N+1) \pi x\big)}{ \sin (\pi x)}. $$
\item Show that for any $\delta > 0$ there exists a constant $c_\delta >0$, such that for any $N \in \mathbb N$ $$ \int\limits_{\delta \le |x| \le 1/2} |D_N (x) | dx \ge c_\delta,$$
i.e. property $(iii)$ in the definition of approximate identities fails.
\item Show that   $$ \| D_N \|_1 \approx \log N,$$ more precisely that there exist constants $c$, $C>0$, such that $ c\log N \le  \| D_N \|_1 \le C \log N$ holds for all $N$ large enough,  i.e. property $(ii)$ in the definition of approximate identities fails.

{\it{Remark:}}  A more precise estimate is $$ \frac{4}{\pi^2} \sum_{k=1}^N \frac1{k} \le \| D_N \|_1 \le 2 + \frac{\pi}{4} +   \frac{4}{\pi^2} \sum_{k=1}^N \frac1{k},$$ but I only ask you to show that $\| D_N \|_1$ grows logarithmically.
\end{enumerate}


\item (Grafakos \#3.1.3) Let $F_N (x)$ be the Fej\'{e}r  kernel, i.e. $$ F_N (x) = \frac1{N+1} \sum_{n=0}^N D_n (x). $$

\begin{enumerate}[(a)]
\item Show that $$ F_N (x)  = \frac{1}{N+1} \left( \frac{\sin \big( (N+1) \pi x\big)}{ \sin (\pi x)} \right)^2 , $$ in particular $F_N (x) \ge 0$. 
\item Show that the family $\{ F_N \}$ is an {\it{approximate identity}} as $N\rightarrow \infty$. 

\end{enumerate}

\item  (Grafakos \#3.1.4) Let $V_N (x)$ be the de la Val\'{e}e Poussin   kernel, i.e. $$ V_N (x) = 2 F_{2N+1} (x) - F_N (x). $$

\begin{enumerate}[(a)]
\item Show that the family $\{ V_N \}$ is an {\it{approximate identity}} as $N\rightarrow \infty$. 
\item Find and plot the Fourier coefficients of $V_N$.
\end{enumerate}
\item  (Grafakos \#3.1.11; Theorem of Fej\'er and F.\~Riesz)  Let $\displaystyle{ P(x) = \sum_{k=-N}^N a_k e^{2\pi i k x} }$ be a trigonometric polynomial such that $ P(x) > 0$ for all $x\in \mathbb T$. Prove that there exists a polynomial  $Q(x)$ of the form  $\displaystyle{ Q(x) = \sum_{k=0}^N b_k e^{2\pi i k x} }$ such that $$P (x) = | Q(x) |^2 . $$

{\it{Hint:}}  Note that $N$ zeros of the complex polynomial   $\displaystyle{ R(z) = \sum_{k=-N}^N a_k z^{k+N} }$ lie inside the unit circle, while the other $N$ lie outside.

\end{enumerate}




 
\end{document}

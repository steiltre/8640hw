%        File: hw2.tex
%     Created: Thu Sep 22 11:00 AM 2016 C
% Last Change: Thu Sep 22 11:00 AM 2016 C
%

\documentclass[a4paper]{article}

\title{Math 8640 Homework 2 }
\date{9/26/16}
\author{Trevor Steil}

\usepackage{amsmath}
\usepackage{amsthm}
\usepackage{amssymb}
\usepackage{esint}

\newtheorem{theorem}{Theorem}[section]
\newtheorem{corollary}{Corollary}[section]
\newtheorem{proposition}{Proposition}[section]
\newtheorem{lemma}{Lemma}[section]
\newtheorem*{claim}{Claim}
\newtheorem*{problem}{Problem}
%\newtheorem*{lemma}{Lemma}
\newtheorem{definition}{Definition}[section]

\newcommand{\R}{\mathbb{R}}
\newcommand{\N}{\mathbb{N}}
\newcommand{\C}{\mathbb{C}}
\newcommand{\Z}{\mathbb{Z}}
\newcommand{\supp}[1]{\mathop{\mathrm{supp}}\left(#1\right)}
\newcommand{\lip}[1]{\mathop{\mathrm{Lip}}\left(#1\right)}
\newcommand{\curl}{\mathrm{curl}}
\newcommand{\la}{\left \langle}
\newcommand{\ra}{\right \rangle}
\renewcommand{\vec}[1]{\mathbf{#1}}

\newenvironment{solution}{\emph{Solution.}}

\begin{document}
\maketitle

\begin{enumerate}

\item Poisson kernel: for $0<r<1$, define the Poisson kernel  as $$P_r (t) = \sum_{n=-\infty}^\infty r^{|n|} e^{2\pi i nt}$$ for $t\in \mathbb T$.

\begin{enumerate}

\item Prove that $\displaystyle{P_r (t) = \textup{Re } \frac{1+ r e^{2\pi i t}}{1- r e^{2\pi i t}} = \frac{1-r^2}{1-2r \cos (2\pi t) + r^2}}$.
\item Deduce that that the family $P_r(t)$ is an approximate identity as $r \rightarrow 1^-$ and observe that $P_r (t)$ is decreasing in $t$ on $[0,1/2)$.
\item Define the conjugate Poisson kernel $$Q_r (t) = -i \sum_{n=-\infty}^\infty \textup{sgn}\, (n) r^{|n|} e^{2\pi i nt}$$ for $t\in \mathbb T$. Show that $\displaystyle{Q_r (t) =  \frac{2r \sin (2\pi t)}{1-2r \cos (2\pi t) + r^2}}$.
\item Let $f\in L^1 (\mathbb T)$ be real-valued. Prove that the function $ z  \rightarrow (P_r \ast f ) (t) + i (Q_r \ast f) (t)$ is analytic in $z = r e^{2\pi i t}$ on the open unit disc $\{ z \in \mathbb C: \, |z|< 1\}$.
\item Conclude that the functions $u (z) = (P_r \ast f ) (t)$ and $v (z) = (Q_r \ast f ) (t)$ are conjugate are conjugate {\it{harmonic}} functions on the open unit disc. In which sense does $f$ represent the boundary value of $u$?

\end{enumerate}

(see Grafakos,  Ex. 3.1.7 and 4.1.4)

\begin{solution}

  \begin{enumerate}
    \item
      \begin{align*}
        P_r(t) &= \sum_{n=-\infty}^\infty r^{|n|} e^{2\pi i n t} \\
        &= \sum_{n=0}^{\infty} r^{n} e^{2 \pi i n t} + \sum_{n=0}^\infty r^n e^{- 2 \pi i n t} - 1 \\
        &= \frac{1}{1-r e^{2\pi i t}} + \frac{1}{1 - r e^{-2 \pi i t}} - 1 \\
        &= \frac{2 - re^{2 \pi i t} - re^{-2 \pi i t}}{1 - r e^{2\pi i t} - re^{-2 \pi i t} + r^2} - 1 \\
        &= \frac{1 - r^2}{1 - 2 r \cos (2 \pi t) + r^2} \\
        &= Re \left( \frac{1 + 2i \sin( 2\pi t) - r^2}{1 - 2 \cos (2 \pi t) + r^2} \right) \\
        &= Re \left( \frac{1 -r e^{-2 \pi i t} + r e^{2 \pi i t} - r^2}{1 - re^{2 \pi i t} - re^{-2 \pi i t} + r^2} \right) \\
        &= Re \left( \frac{1 + r e^{2 \pi i t}}{1 - re^{2\pi i t}} \cdot \frac{1 - re^{-2\pi i t}}{1 - re^{-2\pi i t}} \right) \\
        &= Re \left( \frac{1 + r e^{2 \pi i t}}{1 - re^{2 \pi i t}} \right)
      \end{align*}

    \item
      \begin{align*}
        \int_{0}^{1} P_r(t) dt &= \int_{0}^{1} \sum_{n=-\infty}^\infty r^{|n|} e^{2 \pi i n t} dt \\
        &= \sum_{n = -\infty}^\infty r^{|n|} \int_{0}^{1} e^{2 \pi i n t} dt \quad \text{by Dominated Convergence} \\
        &= 1 \quad \text{by periodicity}
      \end{align*}

      For $r < 1$, we see the numerator in
      \[ P_r(t) = \frac{1 - r^{2}}{1 - 2r \cos(2 \pi t) + r^2} \]
      is positive. The quadratic formula, the roots of the denomiator are
      \begin{equation*}
        \frac{2 \cos(2 \pi t) \pm \sqrt{4( \cos^2(2 \pi t) - 1)}}{2} = \cos(2 \pi t) \pm \sqrt{ - \sin^2 (2 \pi t) }
      \end{equation*}
      which is imaginary, so $P_r(t)$ is positive and $\|P_r(t)\|_{L^1} = 1$ by the calculation above.

      Now let $\delta > 0$. For $t \in [\delta, 1 - \delta]$, $\cos( 2 \pi t )$ is bounded away from $1$, that is, we can find an $M < 1$ such that
      $\cos( 2 \pi t ) \leq M$. Therefore,
      \begin{align*}
        \int_{\delta}^{1 - \delta} P_r(t) dt &= \int_{\delta}^{1-\delta} \frac{1-r^2}{1 - 2r\cos(2 \pi t) + r^2} dt \\
        &\leq \int_{\delta}^{1-\delta} \frac{1-r^2}{1-2M \cdot r + r^2} dt \\
        &\leq \frac{1-r^2}{1 - 2M \cdot r + r^2}
      \end{align*}
      By letting $r \to 1^-$, we see
      \[ \int_{\delta}^{1-\delta} P_r(t) dt = 0 .\]
      Thus, $\{ P_r(t) \}$ is an approximate identity.
      On $[0, \frac{1}{2})$, $\cos(2 \pi t)$ is decreasing, so $P_r(t)$ is as well.

    \item
      We have
      \begin{align*}
        Q_r(t) &= -i \sum_{n=-\infty}^\infty sgn(n) r^{|n|} e^{2 \pi i n t} \\
        &= -i \sum_{n=1}^\infty r^{n} re^{2 \pi i n t} + i \sum_{n=1}^\infty r^{n} re^{-2 \pi i n t} \\
        &= -i \frac{1}{1 - re^{2 \pi i t}} + i \frac{1}{1 - re^{- 2 \pi i t}} \\
        &= -i \frac{re^{2 \pi i t} - re^{-2 \pi i t}}{1 - 2r \cos(2 \pi t) + r^2} \\
        &= \frac{2r \sin(2 \pi t)}{1 - 2r \cos( 2 \pi t ) + r^2}
      \end{align*}

    \item
      We can calculate
      \begin{align*}
        (P_r \ast f)(t) &= \int_{0}^{1} \sum_{n=-\infty}^\infty r^{|n|} e^{2 \pi i n(t-s)} f(s) ds \\
          &= \sum_{n=-\infty}^\infty r^{|n|} e^{2 \pi i n t} \int_{0}^{1} e^{-2 \pi i n s} f(s) ds \quad \text{by Dominated Convergence}\\
          &= \sum_{n=-\infty}^\infty \hat{f}_n r^{|n|} e^{2 \pi i n t}
      \end{align*}

      Similarly,
      \begin{align*}
        i (Q_r \ast f)(t) &= \sum_{n=1}^\infty r^n \hat{f}_n e^{2 \pi i n t} - \sum_{n=1}^\infty r^n \hat{f}_n e^{- 2 \pi i n t}
      \end{align*}
      because of the $sgn(n)$ in $Q_r(t)$.

      Therefore,
      \[ (P_r \ast f)(t) + i (Q_r \ast f)(t) = \sum_{n=1}^\infty \hat{f}_n \left(r e^{2 \pi i t} \right)^n .\]
      This is a convergent power series in $z = re^{2\pi t}$ for $r < 1$ because $\hat{f} \to 0$ by Riemann Lebesgue, and in particular, $\{
        \hat{f}_n\}$ is bounded.

    \item

      The real and imaginary parts of a complex analytic function satisfy the Cauchy-Riemann equations and are therefore conjugate harmonic. Thus,
      $(P_r \ast f)(t)$ and $(Q_r \ast f)(t)$ are conjugate harmonic. $u$ and $f$ agree on the unit circle in the sense that \linebreak \large{\textbf{fill in
      later}}
  \end{enumerate}<++>

\end{solution}

\item (Fej\'er's lemma). Let $f \in L^1 (\mathbb T) $ and $g \in L^\infty (\mathbb T)$. Prove that $$\lim_{n \rightarrow \infty }  \int\limits_{\mathbb T} f(t ) g (nt) \,dt  = \widehat{f} (0) \widehat{g} (0).$$

(See Katznelson, Ch. 1, Sec. 2, Ex. 8)

\begin{proof}

  First, assume $f$ is a trigonometric polynomial. Then we can write $f(t) = \frac{l \in \Z}{c_l} e^{2 \pi i l t}$ and $\hat{f}(0) = c_0$. Define $g_n(t) = g(nt)$. $g \in
  L^{\infty} \supset L^2$, so we can write the Fourier expansion of $g$
  \[ g(t) \sim \sum_{k \in \Z} \hat{g}_k e^{2 \pi i k t} .\]
  Then $g_n$ can be expanded in a Fourier series as
  \begin{align*}
    g_n(t) &= g(nt) \\
    &\sim \sum_{k \in \Z} \hat{g}_k e^{2 \pi i k n t} \\
    &= \sum_{m=kn, k \in \Z} \hat{g} \left(\frac{m}{n} \right) e^{2 \pi i m t}
  \end{align*}
  So we have the Fourier coefficients of $g_n$ are
  \[ \hat{g}_n(m) =
    \begin{cases}
      \hat{g} \left( \frac{m}{n} \right) &\text{if } m = kn \text{ for some } k \in \Z \\
      0 &\text{else}
    \end{cases}
  \]

  We have
  \begin{align*}
    \int_{0}^{1} f(t) g(nt) dt &= \int_{0}^{1} \sum_{l \in \Z} c_l e^{2 \pi i l t} g(nt) dt \\
    &= \sum_{l \in \Z} c_l \int_{0}^{1} e^{2 \pi i l t} g_n(t) dt \quad \text{by Dominated Convergence} \\
    &= \sum_{l \in \Z} c_l \hat{g}_n (l) \\
    &= \sum_{k \in \Z} c_{kn} \hat{g} (k) \\
    &= c_0 \hat{g}(0) + \sum_{k \neq 0} c_{kn} \hat{g}(k) \\
    &\to c_0 \hat{g}(0) \text{ as } n \to \infty
  \end{align*}

  We know $c_0 = \hat{f}(0)$, so the claim is true for trigonometric polynomials.

  Now take an arbitrary $f \in L^1$. We can approximate $f$ by a trigonometric polynomial in the $L^1$ norm. So for any $\varepsilon > 0$, we can find
  a trigonometric polynomial $P(t)$ such that $\|f-P\|_{L^1} < \varepsilon$. Then
  \begin{align*}
    \left| \int_{0}^{1} \left( f(t) - P(t) \right) g(nt) dt \right| &\leq \|g\|_{L^\infty} \|f - P\|_{L^1} \\
    &< \|g\|_{L^\infty} \varepsilon
  \end{align*}

  Letting $\varepsilon \to 0$, we get the result for all $f \in L^{1}$.

\end{proof}

\item  Let $f \in L^1 (\mathbb T)$. Show that if $\sum | \widehat{f} (n) | |n|^{m} < \infty$, then $f$ is $m$ times continuously differentiable.

Deduce that, if $\widehat{f} (n ) = \mathcal O (|n|^{-k})$ for $k>2$ and $$ m = \begin{cases} k-2 \quad k \textup{ integer, } \\ [k]-1 \quad \textup{ otherwise, } \end{cases}$$ then $f$ is $m$ times continuously differentiable.

(See Katznelson, Ch. 1, Sec. 4, Ex. 2)

\begin{proof}

  Let $n = 0$. Define $g_n(x) = \sum_{|k|\leq n} \hat{f}(n) e^{2\pi i k x}$. Notice that each $g_n$ is continuous. We see
  \begin{align*}
    \left| \sum_{k \in \Z} \hat{f}(k) e^{2\pi i k x} - g_n(x) \right| &= \left| \sum_{|k|>n} \hat{f}(k) e^{2 \pi i k x} \right| \\
    &\leq \sum_{|k|>n} |\hat{f}(k)| \\
    &\to 0 \text{ as } n \to \infty
  \end{align*}
  Because this is independent of $x$, we have uniform convergence of $g_n$. $\{ g_n \}$ is a sequence of continuous functions, so $g_n$ converges to a
  continuous function, proving the result for $n=0$.

  Now let $n = 1$. Define $g_n$ as before. By differentiating $g_n$ term-by-term, we get
  \[ g_n'(x) = 2 \pi i \sum_{|k|\leq n} k \hat{f}(k) e^{2 \pi i k x} .\]
  Let $g'(x) = 2 \pi i \sum_{k \in \Z} k \hat{f}(k) e^{2 \pi i k x}$. We see
  \begin{align*}
    \left| g'(x) - g_n'(x) \right| &= \left| 2 \pi i \sum_{|k|>n} k \hat{f}(k) e^{2 \pi i k x} \right| \\
    &\leq 2 \pi \sum_{|k|>n} |k| |\hat{f}_k| \\
    &\to 0 \text{ as } n \to \infty
  \end{align*}

  This is independent of $x$, so $g_n'$ converges uniformly to $g'$. Therefore, $f'(x) = g'(x)$. Also, $\{ g_n' \}$ was a sequence of continuous
  functions, so $f'$ is continuous as well. By applying this result inductively, we get the result for any $m$.

  If $\hat{f}(n) = O(|n|^{-k})$ then
  \[ \hat{f}(n) \leq C |n|^{-k} \]
  for some $C$. Then
  \begin{align*}
    \sum | \hat{f}(n)| |n|^m &\leq C \sum |n|^{m-k} \\
    &\leq \sum |n|^{-1-\varepsilon} \text{ for some } \varepsilon>0 \\
    &< \infty
  \end{align*}
  Therefore, we can apply our result to conclude that $f$ is $m$ times continuously differentiable.

\end{proof}

\item Let $f \in L^1 (\mathbb T)$ and $ \widehat{f} (n ) = \mathcal O (|n|^{-k})$. Show that $f$ is $m$-times  differentiable with $f^{(m)} \in L^2$ provided $k> m +\frac12$.

(See Katznelson, Ch. 1, Sec. 5, Ex. 5)

\begin{proof}

  By problem 3, we know $f$ is $m$-times continuously differentiable. (3 was stated with an equality, but having the inequality only gives us faster
  decay of Fourier coefficients, so it still holds.) All that remains to show is that $f^{(m)} \in L^2$.
  \begin{align*}
    \|f^{(m)}\|_2 &= \left\| \widehat{f^{(m)}} \right\|_2 \\
    &= \sum_{l \in \Z} \left| (2 \pi i l)^m \hat{f}(l) \right|^2 \\
    &\leq (2 \pi)^{2m} \sum_{l \in \Z} |l|^2m \hat{f}^{2}(l) \\
    &\leq (2 \pi)^{2m} \sum_{l \in \Z} C |l|^{2m} |l|^{-2k} \text{ because } \hat{f}(n) = O(|n|^{-k}) \\
    &= C^2 (2 \pi)^{2m} \sum_{l \in \Z} |l|^{2m-2k}
  \end{align*}

  Because $k > m + \frac{1}{2}$, $2m - 2k < -1$. Therefore, the series above is finite, and $f^{(m)} \in L^2$.

\end{proof}

\item Katznelson: Chapter 1, Section 3, Exercise 1 (see this section posted in moodle): Let $f \in L^ 1(\mathbb T)$ and let $0< \alpha \le1$. Assume that $f$ satisfies Lipschitz (H\"older) condition of order $\alpha$ at the point $t_0$. Prove that
$$ \big| \sigma_N f (t_0)   - f(t_0) \big| \le C N^{-\alpha} \,\,\,\,\, \textup{ for } \,\, \alpha<1,$$
and
$$ \big| \sigma_N f (t_0)   - f(t_0) \big| \le C \frac{\log N}{N} \,\,\,\,\, \textup{ for } \,\, \alpha=1,$$
where $\sigma_N f $ is the $N^{th}$ Fej\'er (Ces\`aro) mean of the Fourier series of $f$.

\begin{proof}

  Because $f$ is Lipschitz, we know $|f(t_0 + \tau) -f(t_0)| < C |\tau|^\alpha$ for some $C$. Then we have
  \begin{align*}
    | \sigma_N f(t_0) - f(t_0) | &= \left| F_N \ast f(t_0) - f(t_0) \right| \\
    &= \left| \int_{0}^{1} F_N(t) \left( f(t_0 - t) - f(t_0) \right) dt \right| \\
    &\leq C \int_{0}^{1} | F_N(t) | |t|^\alpha dt \\
    &= C \int_{|t|< \frac{1}{N}}^{} |F_N(t)| |t|^\alpha dt + C \int_{|t|>\frac{1}{N}}^{} |F_N(t)| |t|^\alpha dt
  \end{align*}

  By (3.10) in Katznelson 1.3, we know $F_N(t) \leq \min \{ N+1, \frac{\pi^2}{(N+1)t^2} \}$. Therefore,
  \begin{align*}
    \int_{|t|<\frac{1}{N}}^{} |F_N(t)| |t|^\alpha dt \leq (N+1) \int_{|t|<\frac{1}{N}}^{} |t|^\alpha dt \\
    \leq C |N|^{-\alpha}
  \end{align*}
  for some constant $C$. Assume $\alpha \neq 0$. Then
  \begin{align*}
    \int_{|t|>\frac{1}{N}}^{} |F_N(t)||t|^\alpha dt &\leq \frac{\pi^2}{N+1} \int_{|t|>\frac{1}{N}}^{} |t|^{\alpha - 2} dt \\
    &\leq C |N|^{-\alpha}
  \end{align*}
  for some constant $C$. Combining these two pieces, we have the result for $\alpha < 1$.

  Now let $\alpha = 1$. Then we have
  \begin{align*}
    \int_{|t|>\frac{1}{N}}^{} |F_N(t)||t| dt &\leq \frac{\pi^2}{N+1} \int_{|t|>\frac{1}{N}}^{} |t|^{-1} dt \\
    &\leq C \frac{\log N}{N}
  \end{align*}
  for some constant $C$. Because $\frac{1}{N} = O \left( \frac{\log N}{N} \right)$ for $N>1$, we can combine the two integrals to get the result for
  $\alpha = 1$.

\end{proof}

\end{enumerate}



\end{document}

%        File: hw2.tex
%     Created: Thu Sep 22 11:00 AM 2016 C
% Last Change: Thu Sep 22 11:00 AM 2016 C
%

\documentclass[a4paper]{article}

\title{Math 8640 Homework 2 }
\date{9/26/16}
\author{Trevor Steil}

\usepackage{amsmath}
\usepackage{amsthm}
\usepackage{amssymb}
\usepackage{esint}

\newtheorem{theorem}{Theorem}[section]
\newtheorem{corollary}{Corollary}[section]
\newtheorem{proposition}{Proposition}[section]
\newtheorem{lemma}{Lemma}[section]
\newtheorem*{claim}{Claim}
\newtheorem*{problem}{Problem}
%\newtheorem*{lemma}{Lemma}
\newtheorem{definition}{Definition}[section]

\newcommand{\R}{\mathbb{R}}
\newcommand{\N}{\mathbb{N}}
\newcommand{\C}{\mathbb{C}}
\newcommand{\Z}{\mathbb{Z}}
\newcommand{\supp}[1]{\mathop{\mathrm{supp}}\left(#1\right)}
\newcommand{\lip}[1]{\mathop{\mathrm{Lip}}\left(#1\right)}
\newcommand{\curl}{\mathrm{curl}}
\newcommand{\la}{\left \langle}
\newcommand{\ra}{\right \rangle}
\renewcommand{\vec}[1]{\mathbf{#1}}

\newenvironment{solution}{\emph{Solution.}}

\begin{document}
\maketitle

\begin{enumerate}

\item Poisson kernel: for $0<r<1$, define the Poisson kernel  as $$P_r (t) = \sum_{n=-\infty}^\infty r^{|n|} e^{2\pi i nt}$$ for $t\in \mathbb T$.

\begin{enumerate}

\item Prove that $\displaystyle{P_r (t) = \textup{Re } \frac{1+ r e^{2\pi i t}}{1- r e^{2\pi i t}} = \frac{1-r^2}{1-2r \cos (2\pi t) + r^2}}$.
\item Deduce that that the family $P_r(t)$ is an approximate identity as $r \rightarrow 1^-$ and observe that $P_r (t)$ is decreasing in $t$ on $[0,1/2)$.
\item Define the conjugate Poisson kernel $$Q_r (t) = -i \sum_{n=-\infty}^\infty \textup{sgn}\, (n) r^{|n|} e^{2\pi i nt}$$ for $t\in \mathbb T$. Show that $\displaystyle{Q_r (t) =  \frac{2r \sin (2\pi t)}{1-2r \cos (2\pi t) + r^2}}$.
\item Let $f\in L^1 (\mathbb T)$ be real-valued. Prove that the function $ z  \rightarrow (P_r \ast f ) (t) + i (Q_r \ast f) (t)$ is analytic in $z = r e^{2\pi i t}$ on the open unit disc $\{ z \in \mathbb C: \, |z|< 1\}$.
\item Conclude that the functions $u (z) = (P_r \ast f ) (t)$ and $v (z) = (Q_r \ast f ) (t)$ are conjugate are conjugate {\it{harmonic}} functions on the open unit disc. In which sense does $f$ represent the boundary value of $u$?

\end{enumerate}

(see Grafakos,  Ex. 3.1.7 and 4.1.4)

\begin{solution}

  \begin{enumerate}
    \item
      \begin{align*}
        P_r(t) &= \sum_{n=-\infty}^\infty r^{|n|} e^{2\pi i n t} \\
        &= \sum_{n=0}^{\infty} r^{n} e^{2 \pi i n t} + \sum_{n=0}^\infty r^n e^{- 2 \pi i n t} - 1 \\
        &= \frac{1}{1-r e^{2\pi i t}} + \frac{1}{1 - r e^{-2 \pi i t}} - 1 \\
        &= \frac{2 - re^{2 \pi i t} - re^{-2 \pi i t}}{1 - r e^{2\pi i t} - re^{-2 \pi i t} + r^2} - 1 \\
        &= \frac{1 - r^2}{1 - 2 r \cos (2 \pi t) + r^2} \\
        &= Re \left( \frac{1 + 2i \sin( 2\pi t) - r^2}{1 - 2 \cos (2 \pi t) + r^2} \right) \\
        &= Re \left( \frac{1 -r e^{-2 \pi i t} + r e^{2 \pi i t} - r^2}{1 - re^{2 \pi i t} - re^{-2 \pi i t} + r^2} \right) \\
        &= Re \left( \frac{1 + r e^{2 \pi i t}}{1 - re^{2\pi i t}} \cdot \frac{1 - re^{-2\pi i t}}{1 - re^{-2\pi i t}} \right) \\
        &= Re \left( \frac{1 + r e^{2 \pi i t}}{1 - re^{2 \pi i t}} \right)
      \end{align*}

    \item
      \begin{align*}
        \int_{0}^{1} P_r(t) dt &= \int_{0}^{1} \sum_{n=-\infty}^\infty r^{|n|} e^{2 \pi i n t} dt \\
        &= \sum_{n = -\infty}^\infty r^{|n|} \int_{0}^{1} e^{2 \pi i n t} dt \quad \text{by Dominated Convergence} \\
        &= 1 \quad \text{by periodicity}
      \end{align*}

      For $r < 1$, we see the numerator in
      \[ P_r(t) = \frac{1 - r^{2}}{1 - 2r \cos(2 \pi t) + r^2} \]
      is positive. The quadratic formula, the roots of the denomiator are
      \begin{equation*}
        \frac{2 \cos(2 \pi t) \pm \sqrt{4( \cos^2(2 \pi t) - 1)}}{2} = \cos(2 \pi t) \pm \sqrt{ - \sin^2 (2 \pi t) }
      \end{equation*}
      which is imaginary, so $P_r(t)$ is positive and $\|P_r(t)\|_{L^1} = 1$ by the calculation above.

      Now let $\delta > 0$. For $t \in [\delta, 1 - \delta]$, $\cos( 2 \pi t )$ is bounded away from $1$, that is, we can find an $M < 1$ such that
      $\cos( 2 \pi t ) \leq M$. Therefore,
      \begin{align*}
        \int_{\delta}^{1 - \delta} P_r(t) dt &= \int_{\delta}^{1-\delta} \frac{1-r^2}{1 - 2r\cos(2 \pi t) + r^2} dt \\
        &\leq \int_{\delta}^{1-\delta} \frac{1-r^2}{1-2M \cdot r + r^2} dt \\
        &\leq \frac{1-r^2}{1 - 2M \cdot r + r^2}
      \end{align*}
      By letting $r \to 1^-$, we see
      \[ \int_{\delta}^{1-\delta} P_r(t) dt = 0 .\]
      Thus, $\{ P_r(t) \}$ is an approximate identity.
      On $[0, \frac{1}{2})$, $\cos(2 \pi t)$ is decreasing, so $P_r(t)$ is as well.

    \item
      We have
      \begin{align*}
        Q_r(t) &= -i \sum_{n=-\infty}^\infty sgn(n) r^{|n|} e^{2 \pi i n t} \\
        &= -i \sum_{n=1}^\infty r^{n} re^{2 \pi i n t} + i \sum_{n=1}^\infty r^{n} re^{-2 \pi i n t} \\
        &= -i \frac{1}{1 - re^{2 \pi i t}} + i \frac{1}{1 - re^{- 2 \pi i t}} \\
        &= -i \frac{re^{2 \pi i t} - re^{-2 \pi i t}}{1 - 2r \cos(2 \pi t) + r^2} \\
        &= \frac{2r \sin(2 \pi t)}{1 - 2r \cos( 2 \pi t ) + r^2}
      \end{align*}

    \item
      We can calculate
      \begin{align*}
        (P_r \ast f)(t) &= \int_{0}^{1} \sum_{n=-\infty}^\infty r^{|n|} e^{2 \pi i n(t-s)} f(s) ds \\
          &= \sum_{n=-\infty}^\infty r^{|n|} e^{2 \pi i n t} \int_{0}^{1} e^{-2 \pi i n s} f(s) ds \quad \text{by Dominated Convergence}\\
          &= \sum_{n=-\infty}^\infty \hat{f}_n r^{|n|} e^{2 \pi i n t}
      \end{align*}

      Similarly,
      \begin{align*}
        i (Q_r \ast f)(t) &= \sum_{n=1}^\infty r^n \hat{f}_n e^{2 \pi i n t} - \sum_{n=1}^\infty r^n \hat{f}_n e^{- 2 \pi i n t}
      \end{align*}
      because of the $sgn(n)$ in $Q_r(t)$.

      Therefore,
      \[ (P_r \ast f)(t) + i (Q_r \ast f)(t) = \sum_{n=1}^\infty \hat{f}_n \left(r e^{2 \pi i t} \right)^n .\]
      This is a convergent power series in $z = re^{2\pi t}$ for $r < 1$ because $\hat{f} \to 0$ by Riemann Lebesgue, and in particular, $\{
        \hat{f}_n\}$ is bounded.

    \item

      The real and imaginary parts of a complex analytic function satisfy the Cauchy-Riemann equations and are therefore conjugate harmonic. Thus,
      $(P_r \ast f)(t)$ and $(Q_r \ast f)(t)$ are conjugate harmonic. $u$ and $f$ agree on the unit circle in the sense that \linebreak \large{\textbf{fill in
      later}}
  \end{enumerate}<++>

\end{solution}

\item (Fej\'er's lemma). Let $f \in L^1 (\mathbb T) $ and $g \in L^\infty (\mathbb T)$. Prove that $$\lim_{n \rightarrow \infty }  \int\limits_{\mathbb T} f(t ) g (nt) \,dt  = \widehat{f} (0) \widehat{g} (0).$$

(See Katznelson, Ch. 1, Sec. 2, Ex. 8)

\item  Let $f \in L^1 (\mathbb T)$. Show that if $\sum | \widehat{f} (n) | |n|^{-m} < \infty$, then $f$ is $m$ times continuously differentiable.

Deduce that, if $\widehat{f} (n ) = \mathcal O (|n|^{-k})$ for $k>2$ and $$ m = \begin{cases} k-2 \quad k \textup{ integer, } \\ [k]-1 \quad \textup{ otherwise, } \end{cases}$$ then $f$ is $m$ times continuously differentiable.

(See Katznelson, Ch. 1, Sec. 4, Ex. 2)

\item Let $f \in L^1 (\mathbb T)$ and $ \widehat{f} (n ) = \mathcal O (|n|^{-k})$. Show that $f$ is $m$-times  differentiable with $f^{(m)} \in L^2$ provided $k> m +\frac12$.

(See Katznelson, Ch. 1, Sec. 5, Ex. 5)

\item Katznelson: Chapter 1, Section 3, Exercise 1 (see this section posted in moodle): Let $f \in L^ 1(\mathbb T)$ and let $0< \alpha \le1$. Assume that $f$ satisfies Lipschitz (H\"older) condition of order $\alpha$ at the point $t_0$. Prove that
$$ \big| \sigma_N f (t_0)   - f(t_0) \big| \le C N^{-\alpha} \,\,\,\,\, \textup{ for } \,\, \alpha<1,$$
and
$$ \big| \sigma_N f (t_0)   - f(t_0) \big| \le C \frac{\log N}{N} \,\,\,\,\, \textup{ for } \,\, \alpha=1,$$
where $\sigma_N f $ is the $N^{th}$ Fej\'er (Ces\`aro) mean of the Fourier series of $f$.

\end{enumerate}



\end{document}


